\chapter{Charakterystyka algorytmów uczenia maszynowego}
\section{Uczenie nadzorowane (Supervised Learning)}
Uczenie nadzorowane to technika, w której algorytm otrzymuje gotowy zestaw danych
z wcześniej określonymi etykietami i uczy się na ich podstawie predyktować etykiety dla nowych nieznanych danych.
Dane uczące zawierają zmienne wejściowe oraz zmienną predykowaną (etykietę).
Standardowy proces uczenia nadzorowanego obejmuje podział zbioru danych na zbiór
treningowy, walidacyjny oraz testowy. Trening polega na dopasowaniu parametrów modelu
do danych treningowych poprzez minimalizację funkcji straty, która mierzy różnicę
między przewidywaniami modelu a rzeczywistymi etykietami. Po zakończeniu treningu danych
model zostaje poddany ocenie na zbiorze walidacyjnym w celu doboru hiperparametrów oraz monitorowaniu uczenia maszynowego,
aby zapobiec przeuczeniu modelu. Ostateczna ocena odbywa się na zbiorze testowym na danych nieznanych dla modelu i na podstawie tego modelu określana
wartość dokładności, precyzji, recall, F1 (dla klasyfikacji) lub MSE, MAE (dla regresji)\citep{bishop2006,hastie2009}.
\subsection{Regresja liniowa (Linear Regression)}
