\chapter{Zastosowania uczenia maszynowego w procesach wytwórczych}
\section{Wstęp}
Obecnie znajdujemy się w erze Przemysłu 4.0, gdzie technologia i automatyzacja stanowią podstawę procesów wytwórczych.
Przemysł 4.0 integruje cyfrowe technologie, takie jak big data, przetwarzanie w chmurze, sztuczną inteligencję (AI), Internet Rzeczy (IoT),
robotykę oraz uczenie maszynowe, aby stworzyć inteligentne fabryki \citep{schwab2016,kagermann2013}.
Cyber-fizyczne systemy umożliwiają komunikację między maszynami, sensorami i oprogramowaniem, co pozwala na optymalizację
procesów w czasie rzeczywistym oraz podejmowanie decyzji opartych na danych.
Procesy wytwórcze przedsiębiorstw to element przedsiębiorstwa, który jest zmaga się z szeregiem wyzwań. Takimi jak rosnąca złożoność produktów, 
wymagania dotyczące wysokiej jakości przy jednoczesnej optymalizacji kosztów, konieczność elastycznego reagowania na zmieniające się potrzeby 
rynku oraz presja na redukcję przestojów produkcyjnych. Tradycyjne metody zarządzania produkcją, oparte na statycznych modelach i doświadczeniu 
ekspertów, coraz częściej okazują się niewystarczające w obliczu ogromnych ilości danych generowanych przez nowoczesne systemy produkcyjne \citep{zhong2017}.

Jednakże uczenie maszynowe jako jedna z technologii umożliwia wykorzystanie potencjału Przemysłu 4.0. 
Algorytmy uczenia maszynowego, przedstawione w rozdziale 3, znajdują zastosowanie w analizie danych produkcyjnych, 
umożliwiając automatyczną identyfikację wzorców, predykcję stanów przyszłych oraz optymalizację parametrów procesowych 
bez konieczności jawnego programowania reguł decyzyjnych. Dzięki zdolności do uczenia się z danych historycznych i adaptacji 
do zmieniających się warunków, systemy oparte na uczeniu maszynowym mogą wspierać decyzje w czasie rzeczywistym, poprawiając efektywność, 
jakość i niezawodność procesów wytwórczych \citep{wuest2016,cioffi2020}.

Celem tego rozdziału jest przedstawienie różnych zastosowań uczenia maszynowego w procesach wytwórczych.

\section{Konserwacja predykcyjna}

Konserwacja predykcyjna (Predictive Maintenance, PdM) stanowi zaawansowaną strategię utrzymania ruchu, która wykorzystuje algorytmy uczenia maszynowego do przewidywania awarii maszyn i urządzeń produkcyjnych przed ich wystąpieniem, co skutkuje znaczącą minimalizacją nieplanowanych przestojów i optymalizacją kosztów operacyjnych \citep{lee2014,mobley2002}. W przeciwieństwie do konserwacji reaktywnej oraz prewencyjnej, konserwacja predykcyjna umożliwia proaktywne zarządzanie stanem technicznym poprzez ciągłą analizę danych diagnostycznych i wczesne wykrywanie symptomów degradacji \citep{susto2015,carvalho2019}.

Koncepcja konserwacji predykcyjnej opiera się na założeniu, że każda maszyna generuje charakterystyczne sygnały świadczące o jej stanie zdrowia, które mogą być przechwytywane przez system sensoryczny i przetwarzane w czasie rzeczywistym \citep{ran2019}. Dzięki temu przedsiębiorstwa odchodzą od modeli w których eksploatuje się maszynę do awarii oraz wykonuje się konserwację w stałych ostępach czasowych, a przechodzą do modelu w którym konserwuje się w oparciu o stan techniczny, co prowadzi do istotnej poprawy efektywności wykorzystania zasobów i redukcji całkowitego kosztu posiadania \citep{mobley2002,dalzochio2020}.

\vspace{0.5cm}

\subsection{Architektura systemów konserwacji predykcyjnej}

Współczesne systemy konserwacji predykcyjnej składają się z kilku kluczowych warstw technologicznych \citep{ran2019,zhang2019}:

\vspace{0.5cm}
\textbf{Warstwa zbierania danych} obejmuje rozbudowaną infrastrukturę sensoryczną opartą o technologie Internetu Rzeczy (IoT). Czujniki przemysłowe monitorują krytyczne parametry eksploatacyjne, w tym:
\begin{itemize}
    \item Wibracje mechaniczne -- akcelerometry mierzące drgania łożysk, wałów, przekładni i silników
    \item Temperatura -- czujniki termiczne oraz kamery termowizyjne do wykrywania przegrzania komponentów
    \item Ciśnienie -- przetworniki ciśnienia w systemach hydraulicznych i pneumatycznych
    \item Parametry elektryczne -- mierniki prądu, napięcia i mocy czynnej/biernej silników elektrycznych
    \item Parametry akustyczne -- mikrofony przemysłowe do analizy hałasu i ultradźwięków
    \item Parametry procesowe -- przepływ, stężenie substancji, wilgotność, obroty
\end{itemize}

Dane z czujników są przesyłane do warstwy przetwarzania za pomocą protokołów IoT, takich jak MQTT, OPC UA czy CoAP, zapewniając niskie opóźnienia i wysoką przepustowość \citep{zhang2019}.

\vspace{0.5cm}
\textbf{Warstwa edge computing} umożliwia wstępne przetwarzanie danych bezpośrednio w lokalizacji maszyny, co redukuje obciążenie sieci i opóźnienia związane z przesyłem danych do chmury \citep{dalzochio2020}. Urządzenia brzegowe wykonują:
\begin{itemize}
    \item Filtrację i agregację danych surowych
    \item Ekstrakcję cech w dziedzinie czasu i częstotliwości
    \item Detekcję anomalii w czasie rzeczywistym z wykorzystaniem lekkich modeli ML
    \item Kompresję danych przed wysyłką do chmury
\end{itemize}

\textbf{Warstwa analityki i modelowania} wykorzystuje algorytmy uczenia maszynowego do budowy modeli predykcyjnych \citep{carvalho2019,susto2015}. Najczęściej stosowane metody to:
\begin{itemize}
    \item \textbf{Random Forest i Gradient Boosting} -- do klasyfikacji stanów maszyny oraz regresji pozostałego czasu użytkowania
    \item \textbf{Support Vector Machines (SVM)} -- do klasyfikacji typów usterek na podstawie wzorców wibracyjnych
    \item \textbf{Sieci neuronowe LSTM i GRU} -- do analizy sekwencji czasowych i modelowania trajektorii degradacji
\end{itemize}

\textbf{Warstwa integracji systemowej} łączy systemy PdM z istniejącą infrastrukturą IT przedsiębiorstwa, w tym z systemami SCADA (Supervisory Control and Data Acquisition), MES (Manufacturing Execution System) oraz ERP (Enterprise Resource Planning) \citep{zhang2019}. Integracja ta umożliwia automatyczne generowanie zleceń konserwacyjnych, optymalizację harmonogramów napraw oraz śledzenie kosztów i wskaźników efektywności (KPI).

\vspace{0.5cm}

\subsection{Zastosowania branżowe i efekty wdrożeń}

Konserwacja predykcyjna znajduje szerokie zastosowanie w różnych sektorach przemysłu, przynosząc wymierne korzyści operacyjne i finansowe \citep{lee2014,carvalho2019}.

\textbf{Przemysł motoryzacyjny} wykorzystuje PdM do monitorowania linii montażowych i robotów przemysłowych. Nieplanowany przestój linii produkcyjnej w fabryce samochodów może kosztować od 500 tys. do 1 mln USD za godzinę \citep{lee2014}. Implementacja systemów predykcyjnych w fabrykach koncernów takich jak BMW, Volkswagen czy General Motors przyniosła redukcję przestojów o 30-50\% oraz wydłużenie żywotności krytycznych komponentów o 20-25\% \citep{susto2015}.

\textbf{Energetyka} stosuje PdM do turbozespołów elektrowni, turbin wiatrowych i transformatorów. W przypadku turbin wiatrowych, które pracują w trudnych warunkach atmosferycznych i są trudno dostępne, predykcja awarii łożysk i przekładni z wyprzedzeniem umożliwia planowanie napraw w optymalnych warunkach pogodowych \citep{ran2019}. Producenci turbin gazowych raportują osiągnięcie bardzo wysokich poziomów dostępności dzięki systemom predykcyjnym opartym na LSTM i analizie setek parametrów procesowych \citep{lee2014}.

\textbf{Przemysł farmaceutyczny} wymaga wysokiej niezawodności procesów ze względu na wymogi regulacyjne. Systemy PdM monitorują urządzenia w procesach sterylnych, takich jak autoklawy, suszarki próżniowe czy bioreaktory, zapewniając ciągłość produkcji i zgodność z procedurami walidacyjnymi \citep{dalzochio2020}. Redukcja przestojów w produkcji szczepionek czy leków biologicznych o 15-25\% przekłada się na oszczędności rzędu milionów dolarów rocznie.

\textbf{Przemysł elektroniczny} wykorzystuje PdM do utrzymania precyzyjnych systemów pozycjonowania, kontroli atmosfery oraz urządzeń do nanoszenia warstw. W fabrykach półprzewodników, gdzie wartość produktu w toku jest ekstremalna wysoka, nawet krótkie przestoje prowadzą do strat milionowych \citep{susto2015}.

\vspace{0.5cm}

\subsection{Korzyści ekonomiczne i operacyjne}

Wdrożenia systemów konserwacji predykcyjnej generują wielowymiarowe korzyści \citep{mobley2002,carvalho2019}:

\begin{itemize}
    \item \textbf{Redukcja kosztów konserwacji o 10-40\%} -- eliminacja zbędnych przeglądów prewencyjnych oraz optymalizacja zużycia części zamiennych
    \item \textbf{Wzrost dostępności maszyn powyżej 95\%} -- minimalizacja nieplanowanych przestojów i skrócenie czasu napraw dzięki przygotowaniu części i personelu z wyprzedzeniem
    \item \textbf{Przedłużenie żywotności aktywów o 15-30\%} -- optymalne planowanie wymian komponentów przed katastroficznym uszkodzeniem
    \item \textbf{Poprawa bezpieczeństwa personelu} -- wczesne wykrywanie warunków niebezpiecznych (przegrzanie, nadmierne drgania) redukuje ryzyko wypadków
    \item \textbf{Optymalizacja zapasów magazynowych} -- precyzyjna predykcja zapotrzebowania na części zamienne umożliwia redukcję kapitału zamrożonego w magazynie o 20-35\%
    \item \textbf{Wzrost efektywności energetycznej} -- wykrywanie niewydolnych komponentów prowadzi do oszczędności energii rzędu 5-15\%
\end{itemize}

Zwrot z inwestycji (ROI) dla projektów PdM zazwyczaj osiągany jest w horyzoncie 12-24 miesięcy, przy czym krótsze okresy zwrotu charakteryzują branże o wysokich kosztach przestojów (motoryzacja, półprzewodniki, energetyka) \citep{mobley2002}.

\section{Optymalizacja parametrów procesów produkcyjnych}

Optymalizacja parametrów procesów produkcyjnych z wykorzystaniem uczenia maszynowego jest elementem realizacji koncepcji inteligentnej fabryki w ramach idei Przemysłu 4.0 \citep{wang2018,monostori2016}. Tradycyjne podejścia do optymalizacji oparte na statycznych modelach fizycznych, heurystykach lub doświadczeniu inżynierów procesowych są niewystarczające w obliczu rosnącej złożoności procesów, dynamicznie zmieniających się warunków produkcji oraz ogromnych ilości danych generowanych przez współczesne systemy automatyki przemysłowej \citep{kusiak2018}.

Uczenie maszynowe oferuje możliwość dynamicznego dostosowywania parametrów procesowych takich jak prędkość linii, temperatura, ciśnienie, przepływ, dozowanie substancji, siły skrawania czy parametry spawania w czasie rzeczywistym, na podstawie bieżących danych z czujników oraz przewidywanych zmian w środowisku produkcyjnym \citep{sharp2019,weichert2019}. Dzięki zdolności do modelowania złożonych, nieliniowych zależności między parametrami wejściowymi a wskaźnikami jakości, wydajności i kosztów, algorytmy ML umożliwiają osiągnięcie optymalnych punktów pracy procesów, które byłyby trudne lub niemożliwe do znalezienia metodami konwencjonalnymi \citep{wang2018}.

\vspace{0.5cm}

\subsection{Architektura systemów optymalizacji}

Systemy optymalizacji procesów produkcyjnych oparte na uczeniu maszynowym składają się z kilku współpracujących ze sobą modułów \citep{sharp2019,psarommatis2020}:

\textbf{Moduł akwizycji i integracji danych} gromadzi informacje z wielorakich źródeł, obejmujących:
\begin{itemize}
    \item Czujniki IoT na liniach produkcyjnych (mierzące parametry takie jak: temperatura, ciśnienie, wilgotność, przepływ, wibracje, akustyka)
    \item Systemy wizyjne i spektrometryczne do oceny jakości produktów w czasie rzeczywistym
    \item Dane z systemów MES i SCADA -- parametry ustawień maszyn, czasy cyklu, ilości produkcji
    \item Dane z systemów ERP -- zlecenia produkcyjne, dostępność surowców, harmonogramy
    \item Dane środowiskowe -- temperatura i wilgotność otoczenia, jakość zasilania elektrycznego
\end{itemize}

Integracja tych źródeł wymaga zastosowania warstw pośredniczących implementujących standardy komunikacyjne takie jak OPC UA, MQTT, czy DDS (Usługa Dystrybucji Danych) \citep{psarommatis2020}.

\textbf{Moduł przetwarzania w czasie rzeczywistym} wykorzystuje technologie przetwarzania brzegowego do wstępnej analizy danych oraz podejmowania szybkich decyzji optymalizacyjnych bez konieczności komunikacji z centralną chmurą obliczeniową \citep{wang2018}. Minimalizacja opóźnień jest krytyczna dla procesów o wysokiej dynamice, takich jak obróbka skrawaniem, wtrysk plastiku czy formowanie blach, gdzie parametry muszą być dostosowywane w przedziałach czasowych rzędu milisekund do sekund.

\textbf{Moduł modelowania i uczenia} implementuje różnorodne algorytmy uczenia maszynowego dostosowane do specyfiki problemu optymalizacyjnego \citep{kusiak2018,sharp2019}:

\begin{itemize}
    \item \textbf{Uczenie nadzorowane} -- modele regresji prognozują wskaźniki jakości i wydajności na podstawie parametrów procesowych. Wytrenowane na danych historycznych, modele te służą jako modele zastępcze rzeczywistych procesów, umożliwiając szybką symulację różnych scenariuszy bez potrzeby eksperymentów fizycznych
    
    \item \textbf{Uczenie nienadzorowane} -- algorytmy klasteryzacji identyfikują reżimy operacyjnei wykrywają anomalie w zachowaniu procesów.
    
    \item \textbf{Uczenie ze wzmocnieniem} -- agenci uczą się optymalnych polityk sterowania poprzez interakcję z procesem (fizycznym lub symulowanym), maksymalizując kumulatywną nagrodę zdefiniowaną jako funkcja wydajności, jakości i kosztów \citep{wang2018,monostori2016}. Uczenie ze wzmocnieniem jest szczególnie efektywne w problemach sekwencyjnego podejmowania decyzji, gdzie bieżące działania wpływają na przyszłe stany systemu
    
\end{itemize}

\textbf{Moduł sterowania adaptacyjnego} implementuje zamkniętą pętlę sterowania, w której obliczone parametry optymalne są przekazywane do sterowników PLC/DCS linii produkcyjnych \citep{psarommatis2020}. System monitoruje efekty zmian parametrów i dynamicznie dostosowuje strategię w odpowiedzi na odchylenia od prognozowanych wyników.

\vspace{0.5cm}

\subsection{Techniki i algorytmy optymalizacji}

W praktyce przemysłowej stosuje się różne podejścia do optymalizacji, w zależności od charakterystyki procesu i dostępnych danych \citep{sharp2019,wang2018}.

\textbf{Optymalizacja oparta na modelach zastępczych} polega na wytrenowaniu modelu uczenia maszynowego jako symulacji rzeczywistego procesu produkcyjnego \citep{shahriari2016}. Model ten, znacznie szybszy w ewaluacji niż proces fizyczny, jest następnie wykorzystywany w procedurach optymalizacyjnych do poszukiwania optymalnych ustawień parametrów. Podejście to jest szczególnie wartościowe w procesach o długich czasach cyklu (np. obróbka cieplna, utwardzanie kompozytów), gdzie eksperymentacja fizyczna jest kosztowna czasowo.

\textbf{Optymalizacja w czasie rzeczywistym z wykorzystaniem uczenia wzmocnionego} umożliwia adaptacyjne dostosowywanie parametrów w trakcie produkcji \citep{wang2018,monostori2016}. Agent obserwuje stan procesu (wartości czujników, wskaźniki jakości) i podejmuje akcje poprzez zmiane parametrów, otrzymując nagrodę proporcjonalną do osiągniętych wskaźników wydajnościowych.

\textbf{Hybrydowe podejścia łączące wiedzę ekspertową z ML} integrują reguły fizyczne i ograniczenia procesowe z modelami data-driven \citep{weichert2019}. Na przykład, w procesach spawania, model uczenia maszynowego może optymalizować parametry (prąd, napięcie, prędkość) w ramach okna parametrów zdefiniowanego przez normy spawalnicze i właściwości materiałów.

\vspace{0.5cm}

\subsection{Zastosowania branżowe i rezultaty wdrożeń}

Optymalizacja oparta o uczenie maszynowe znajduje zastosowanie w szerokim spektrum procesów produkcyjnych \citep{wang2018,kusiak2018,sharp2019}.

\textbf{Przemysł motoryzacyjny} wykorzystuje uczenie maszynowe do optymalizacji linii montażowych, lakierowania i procesów spawalniczych. W procesie lakierowania, algorytmy optymalizują natężenie przepływu lakieru, ciśnienie w komorach, temperaturę i wilgotność w celu minimalizacji zużycia materiału przy zachowaniu jednolitej grubości powłoki \citep{wang2018}. Producenci motoryzacyjni raportują znaczącą redukcję odpadów produkcyjnych oraz istotną poprawę efektywności energetycznej linii produkcyjnych po wdrożeniu systemów optymalizacji opartych na uczeniu ze wzmocnieniem.

\textbf{Przemysł spożywczy i opakowaniowy} stosuje uczenie maszynowe do optymalizacji procesów mieszania, formowania, pieczenia i pakowania. W produkcji kopert i opakowań kartonowych, systemy analizują jakość surowca (gramatura, wilgotność papieru) i dynamicznie dostosowują prędkość linii, siły dociskowe i temperatury klejenia \citep{kusiak2018}. Producenci opakowań raportują znaczącą redukcję wadliwych produktów po wdrożeniu optymalizacji ML w liniach aseptycznego pakowania.

\textbf{Przemysł chemiczny i petrochemiczny} wykorzystuje uczenie maszynowe do optymalizacji reaktorów, kolumn destylacyjnych i procesów separacji. W rafineriach, modele predykcyjne optymalizują temperatury, ciśnienia i przepływy w kolumnach destylacyjnych w celu maksymalizacji uzysku frakcji wartościowych (benzyna, olej napędowy) przy minimalizacji zużycia energii \citep{wang2018}. Koncerny petrochemiczne raportują wzrost wydajności procesów oraz redukcję emisji \ce{CO2} dzięki systemom opartym na uczeniu maszynowym.

\textbf{Produkcja półprzewodników} wymaga ekstremalnej precyzji parametrów procesowych. Uczenie maszynowe optymalizuje procesy litografii, trawienia i nanoszenia warstw, gdzie nawet minimalne odchylenia parametrów prowadzą do defektów kosztujących miliony dolarów \citep{sharp2019}. Producenci półprzewodników wykorzystują systemy optymalizacji do utrzymania wysokiej wydajności produkcyjnej dla najbardziej zaawansowanych węzłów technologicznych.

\textbf{Przemysł tekstylny i odzieżowy} stosuje uczenie maszynowe do optymalizacji procesów farbowania, wykańczania i krojenia tkanin. Systemy optymalizują zużycie wody, barwników i energii w procesach farbowania \citep{kusiak2018}.

\section{Kontrola jakości i analiza predykcyjna}

Kontrola jakości stanowi kluczowy element zarządzania procesami wytwórczymi, bezpośrednio wpływając na satysfakcję klientów, konkurencyjność produktów oraz rentowność przedsiębiorstwa \citep{huang2021,wang2020qc}. Tradycyjne metody kontroli jakości oparte na manualnej inspekcji, próbkowaniu statystycznym oraz kartach kontrolnych charakteryzują się istotnymi ograniczeniami: są czasochłonne, subiektywne, kosztowne i często wykrywają defekty, gdy wadliwe produkty już zostały wyprodukowane \citep{psarommatis2019}.
Uczenie maszynowe rewolucjonizuje podejście do kontroli jakości poprzez umożliwienie automatycznego wykrywania defektów w czasie rzeczywistym, analizy predykcyjnej trendów jakościowych oraz identyfikacji przyczyn źródłowych problemów jakościowych \citep{weimer2016,wang2020qc}. Algorytmy uczenia maszynowego, w szczególności techniki wizji komputerowej oparte na sieciach neuronowych konwolucyjnych, potrafią analizować obrazy produktów z dokładnością przewyższającą ludzkich inspektorów, wykrywając subtelne defekty niedostrzegalne gołym okiem \citep{tabernik2020,yang2020}.

Analiza predykcyjna wykorzystuje dane historyczne z procesów produkcyjnych parametry maszyn, właściwości materiałów, warunki środowiskowe, wyniki testów do prognozowania przyszłej jakości produktów przed ich faktycznym wytworzeniem \citep{huang2021}. Dzięki temu możliwe jest proaktywne zapobieganie problemom jakościowym poprzez wcześniejszą interwencję i korektę parametrów procesowych, minimalizując odpady, przeróbki i straty finansowe.

\vspace{0.5cm}

\subsection{Automatyczna inspekcja wizualna oparta na uczeniu głębokim}

Wizja komputerowa wspomagana przez sieci neuronowe stanowi najbardziej zaawansowaną technologię w automatycznej kontroli jakości produktów \citep{tabernik2020}.

\textbf{Architektury sieciowe dla inspekcji wizualnej}

Sieci konwolucyjne (CNN) stanowią fundament systemów inspekcji wizualnej \citep{tabernik2020}. Podstawowe architektury obejmują:

\begin{itemize}
    \item \textbf{Klasyfikacja defektów} -- sieci takie jak ResNet, EfficientNet czy Vision Transformer klasyfikują obrazy na kategorie lub rozpoznają typy defektów. Osiągają dokładność klasyfikacji powyżej 95-99\% dla dobrze zdefiniowanych klas defektów \citep{yang2020}
    
    \item \textbf{Segmentacja semantyczna} -- architektury typu U-Net, DeepLab czy Mask R-CNN lokalizują defekty na poziomie pikseli, tworząc precyzyjne maski obszarów wadliwych. Umożliwia to nie tylko wykrycie defektu, ale również jego wymiarowanie i lokalizację przestrzenną \citep{tabernik2020}
    
    \item \textbf{Detekcja obiektów} -- algorytmy takie jak YOLO, Faster R-CNN czy RetinaNet wykrywają i lokalizują wiele defektów jednocześnie w czasie rzeczywistym (>30 klatek na sekundę), co jest ważne, gdy praca jest wykonywana na szybkiej linii produkcyjnej \citep{wang2020qc}
    
    \item \textbf{Few-shot learning i transfer learning} -- w przemysłowych zastosowaniach często brakuje dużych zbiorów etykietowanych obrazów defektów. Techniki takie jak few-shot learning, meta-learning czy transfer learning z modeli pretrenowanych na ImageNet umożliwiają efektywne uczenie przy ograniczonej liczbie przykładów (nawet 10-50 obrazów na klasę) \citep{weimer2016}
\end{itemize}

\textbf{Detekcja anomalii bez nadzoru}

W wielu scenariuszach przemysłowych uzyskanie etykietowanych przykładów defektów jest trudne lub niemożliwe ze względu na ich rzadkość i różnorodność. Metody uczenia nienadzorowanego adresują ten problem \citep{bergmann2020,tabernik2020}:

\begin{itemize}
    \item \textbf{Autoencodery} -- sieci neuronowe trenowane do rekonstrukcji obrazów produktów prawidłowych. Podczas inspekcji, obrazy z defektami generują wysokie błędy rekonstrukcji                       , co umożliwia ich wykrycie bez konieczności posiadania etykietowanych przykładów defektów
    
    \item \textbf{Generatywne sieci przeciwstawne (GANs)} -- modele takie jak AnoGAN czy f-AnoGAN uczą się rozkładu prawdopodobieństwa obrazów prawidłowych, a następnie wykrywają anomalie jako punkty o niskim prawdopodobieństwie
    
    \item \textbf{Jednoklasowy SVM i Las Izolacyjny (One-class SVM i Isolation Forest)} -- klasyczne metody ML stosowane do detekcji outlierów w przestrzeni cech wyekstrahowanych przez CNN
    
    \item \textbf{Uczenie pod własnym nadzorem (Self-supervised learning)} -- techniki takie jak SimCLR, MoCo czy DINO pozwalają na uczenie reprezentacji wizualnych bez etykiet, które następnie mogą być wykorzystane do detekcji anomalii
\end{itemize}

\textbf{Implementacja systemów inspekcji wizualnej}

Typowy system automatycznej inspekcji wizualnej składa się z następujących komponentów \citep{yang2020,wang2020qc}:

\begin{itemize}
    \item \textbf{Kamery przemysłowe} -- wysokorozdzielcze kamery (5-20 Mpx) z częstotliwością próbkowania 30-200 FPS, wyposażone w obiektywy telcentryczne lub standardowe.
    \item \textbf{Oświetlenie} -- element warunkujący jakość obrazów. Przy ustawianiu odpowiedniego oświetlenia trzeba zadbać o równomierne rozproszenie, podświetlenie dla przezroczystych obiektów, eliminacja cieni, pomiary 3D.
    \item \textbf{Urządzenia do przetwarzania brzegowego (Edge computing devices)} -- jednostki GPU (NVIDIA Jetson, Intel Neural Compute Stick) lub specjalizowane akceleratory AI (Google Edge TPU, Intel Movidius) realizujące inferencję modeli w czasie rzeczywistym bezpośrednio na linii produkcyjnej.
    \item \textbf{Mechanizmy odrzutu} -- systemy pneumatyczne lub mechaniczne automatycznie usuwające wadliwe produkty z linii na podstawie wyników inspekcji.
    \item \textbf{Interfejs operator-system} -- dashboardy wizualizujące statystyki defektów, trendy jakościowe, alerty oraz umożliwiające weryfikację decyzji systemu przez operatorów.
\end{itemize}

Nowoczesne systemy osiągają przepustowość inspekcji rzędu 100-1000 produktów/minutę przy dokładności wykrywania defektów powyżej 95\% i współczynniku fałszywych alarmów poniżej 2-5\% \citep{tabernik2020}.

\vspace{0.5cm}

\subsection{Analiza predykcyjna jakości}

Predykcja jakości produktów na podstawie danych procesowych umożliwia proaktywne zarządzanie jakością i minimalizację strat produkcyjnych \citep{psarommatis2019}.

\textbf{Modele predykcyjne i źródła danych}

Systemy analityki predykcyjnej jakości (Predictive Quality Analytics) budują modele statystyczne i uczenia maszynowego mapujące relacje między parametrami wejściowymi procesów a wskaźnikami jakości produktów \citep{huang2021}. Źródła danych obejmują:

\begin{itemize}
    \item Parametry procesowe -- temperatura, ciśnienie, prędkość, przepływ, siły, napięcia, pobór prądu
    \item Właściwości materiałów -- gramatura, wilgotność, skład chemiczny, twardość, ciągliwość
    \item Warunki środowiskowe -- temperatura i wilgotność hali produkcyjnej, jakość zasilania elektrycznego
    \item Dane o urządzeniach -- wiek, stopień zużycia, historia konserwacji
    \item Historyczne dane jakościowe -- wyniki testów laboratoryjnych, pomiary wymiarowe, wyniki inspekcji
\end{itemize}

Algorytmy stosowane w PQA to m.in. Random Forest, Gradient Boosting (XGBoost, LightGBM), sieci neuronowe (MLP, LSTM dla danych sekwencyjnych), SVM oraz modele liniowe (ridge regression, LASSO) dla interpretowalności \citep{wang2020qc}.

\subsection{Zastosowania branżowe}

Systemy kontroli jakości bazujacej na uczeniu maszynowym znajdują zastosowanie w wielu sektorach przemysłowych.

\textbf{Przemysł motoryzacyjny} -- inspekcja wizualna spawów, powłok lakierniczych, montażu komponentów oraz testowanie funkcjonalne. Systemy CNN wykrywają mikrorysy w lakierze, nierównomierności spoin, błędy montażu. Redukcja defektów dochodzi do 30-50\%, a koszty inspekcji spadają o 40-60\% w porównaniu z inspekcją manualną \citep{weimer2016,yang2020}. BMW, Audi i Tesla wdrożyły systemy inspekcji AI na liniach karoserii i montażu końcowego.

\textbf{Przemysł spożywczy} -- kontrola kształtu, koloru, rozmiaru, obecności ciał obcych w produktach spożywczych. Systemy analizują owoce, warzywa, pieczywo, mięso, produkty mleczarskie. W produkcji jogurtów kontrola polega na detekcji wad opakowań, błędnych etykiet, nieszczelności. W przetwórstwie mięsnym polega na wykrywanie fragmentów kości, tworzy sztucznych, kontaminacji \citep{yang2020}. Systemy zapewniają zgodność z normami BRC, IFS, FSSC 22000.

\textbf{Przemysł farmaceutyczny} -- inspekcja tabletek podkątem wad wizualnych, ampułek podkątem uszkodzeń, blistrów pod kątem kompletności i jakości wizualnej tabletek. Systemy muszą spełniać wymogi FDA 21 CFR Part 11 i EU Annex 11 \citep{wang2020qc}. Pfizer, Novartis i Roche stosują systemy inspekcji AI z wymaganą wykrywalnością defektów krytycznych >99.9\%.

\textbf{Przemysł tekstylny} -- detekcja wad tkanin w czasie rzeczywistym na maszynach tkackich. Systemy CNN przetwarzają obrazy z kamer liniowych przy prędkości tkaniny 50-100 m/min \citep{tabernik2020}.

\textbf{Produkcja stali i metali} -- inspekcja powierzchni blach, prętów, rur. Wykrywanie wad powierzchniowych, pomiary wymiarowe, kontrola powłok antykorozyjnych \citep{yang2020}.
