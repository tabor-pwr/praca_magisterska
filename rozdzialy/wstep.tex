\chapter{Wstęp}

\section{Kontekst i motywacja}

Współczesne procesy wytwórcze stoją przed wyzwaniami takimi jak: rosnącą złożonością produktów, globalizacją łańcuchów dostaw, coraz wyższymi wymaganiami jakościowymi oraz presją na optymalizację kosztów. Tradycyjne podejścia zarządcze, oparte głównie na doświadczeniu ekspertów i statycznych modelach, okazują się coraz mniej efektywne. Jednocześnie nowoczesne systemy produkcyjne generują ogromne ilości danych z czujników, systemów monitorowania i kontroli jakości, które często pozostają niewykorzystane.

Przemysł 4.0 przynosi nowe możliwości wykorzystania tych danych poprzez uczenie maszynowe. Ta technologia umożliwia automatyczne wykrywanie wzorców w danych, budowanie modeli predykcyjnych oraz podejmowanie decyzji w czasie rzeczywistym. Zastosowanie algorytmów uczenia maszynowego w procesach wytwórczych otwiera perspektywy znaczącej poprawy efektywności, jakości produktów i niezawodności linii produkcyjnych.

Potencjał uczenia maszynowego w przemyśle jest szeroki: od konserwacji predykcyjnej, przez automatyczną kontrolę jakości z wykorzystaniem wizji komputerowej, po optymalizację parametrów procesowych. Pomimo rosnącego zainteresowania tematyką, praktyczne wdrożenia w polskim przemyśle są wciąż rzadkie, co wynika z barier kompetencyjnych, kulturowych i organizacyjnych.

\section{Cel i zakres pracy}

Celem pracy jest kompleksowe przedstawienie uczenia maszynowego jako narzędzia doskonalenia procesów wytwórczych, ze szczególnym uwzględnieniem teoretycznych podstaw, charakterystyki algorytmów oraz praktycznych zastosowań.

Praca realizuje następujące cele szczegółowe:
\begin{enumerate}
    \item Przegląd literatury naukowej dotyczącej uczenia maszynowego i jego zastosowań w procesach wytwórczych, z uwzględnieniem fundamentalnych prac teoretycznych oraz badań nad implementacją w kontekście Przemysłu 4.0.
    \item Charakterystyka podstawowych algorytmów uczenia maszynowego. Metod uczenia nadzorowanego, nienadzorowanego oraz uczenia przez wzmacnianie, wraz z omówieniem ich właściwości, zalet i ograniczeń.
    \item Analiza zastosowań uczenia maszynowego w kluczowych obszarach procesów wytwórczych: konserwacja predykcyjna, kontrola jakości, optymalizacja procesów, prognozowanie popytu.
    \item Przeprowadzenie studium przypadku poprzez wytrenowanie i porównanie modeli uczenia maszynowego do klasyfikacji defektów w produktach odlewniczych jako praktyczna weryfikacja omówionych metod.
\end{enumerate}

Zakres pracy obejmuje zarówno aspekty teoretyczne, jak i praktyczne. Część teoretyczna dostarcza fundamentu metodologicznego, część praktyczna w postaci studium przypadku demonstruje pełny cykl projektu uczenia maszynowego, od przygotowania danych, przez trening modeli, aż po ewaluację wyników.

\section{Metodyka badawcza}

Przyjęta metodyka łączy przegląd literaturowy z eksperymentem obliczeniowym. Część teoretyczna opiera się na analizie klasycznych publikacji z zakresu uczenia maszynowego oraz współczesnych badań nad zastosowaniami przemysłowymi.

Część empiryczna wykorzystuje rzeczywisty zbiór danych zawierający obrazy produktów odlewniczych z oznaczeniem obecności lub braku defektów. Zaimplementowano sześć algorytmów klasyfikacji: drzewa decyzyjne, lasy losowe, maszyny wektorów nośnych, sieci neuronowe, k-najbliższych sąsiadów oraz regresję logistyczną. Wszystkie modele wytrenowano na tym samym zbiorze z jednolitą metodyką podziału 80:20 na zbiór treningowy i testowy. Ocena obejmuje standardowe metryki: dokładność, precyzję, czułość, miarę F1 oraz czas treningu.


\section{Oczekiwane rezultaty}

Praca ma dostarczyć kompleksowego spojrzenia na uczenie maszynowe jako narzędzie doskonalenia procesów wytwórczych. Oczekiwanym rezultatem jest nie tylko przedstawienie stanu wiedzy i charakterystyki algorytmów, ale także konkretny przykład skutecznej implementacji. Poprzez identyfikację możliwości i ograniczeń poszczególnych metod, praca powinna przyczynić się do bardziej świadomego wykorzystania potencjału uczenia maszynowego w optymalizacji procesów produkcyjnych.