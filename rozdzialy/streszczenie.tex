\chapter*{Streszczenie}
\addcontentsline{toc}{chapter}{Streszczenie}

Niniejsza praca magisterska przedstawia kompleksową analizę uczenia maszynowego jako narzędzia doskonalenia procesów wytwórczych w erze Przemysłu 4.0. Praca obejmuje przegląd literatury, szczegółową charakterystykę algorytmów oraz studium przypadku klasyfikacji defektów w produktach odlewniczych. Zaimplementowano i porównano sześć algorytmów: drzewa decyzyjne, lasy losowe, maszyny wektorów nośnych, sieci neuronowe, k-najbliższych sąsiadów oraz regresję logistyczną na zbiorze 1300 obrazów. Random Forest osiągnął najwyższą skuteczność (91,15\% dokładności) przy krótkim czasie treningu (2,31 s). Praca dowodzi praktycznej przydatności uczenia maszynowego w automatycznej kontroli jakości i wskazuje kierunki dalszego rozwoju, w tym zastosowanie głębokiego uczenia i wdrożenia pilotażowe.

\vspace{1cm}

\noindent\textbf{Słowa kluczowe:} uczenie maszynowe, procesy wytwórcze, Przemysł 4.0, klasyfikacja defektów, wizja komputerowa, Random Forest

\vspace{2cm}

\chapter*{Abstract}
\addcontentsline{toc}{chapter}{Abstract}

This master's thesis presents a comprehensive analysis of machine learning as a tool for improving manufacturing processes in the Industry 4.0 era. The work includes a literature review, detailed characterization of algorithms, and a case study on defect classification in casting products. Six algorithms were implemented and compared: decision trees, random forests, support vector machines, neural networks, k-nearest neighbors, and logistic regression on a dataset of 1300 images. Random Forest achieved the highest effectiveness (91.15\% accuracy) with short training time (2.31 s). The thesis demonstrates the practical utility of machine learning in automated quality control and indicates directions for further development, including deep learning applications and pilot implementations.

\vspace{1cm}

\noindent\textbf{Keywords:} machine learning, manufacturing processes, Industry 4.0, defect classification, computer vision, Random Forest
